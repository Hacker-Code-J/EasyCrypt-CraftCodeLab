%\documentclass[tikz,border=3mm]{standalone}
%\usepackage{amsmath}
%\usetikzlibrary{arrows.meta, positioning, decorations.pathmorphing}
%
%\begin{document}
%	\begin{tikzpicture}[
%		node distance=2.5cm and 2.5cm,
%		box/.style={draw, rounded corners, minimum height=1cm, minimum width=2cm, align=center},
%		arrow/.style={-Stealth, thick},
%		sample/.style={dashed, -Stealth, thick},
%		indist/.style={<->, thick, decorate, decoration={snake, amplitude=1mm, segment length=2mm}}
%		]
%		% Nodes
%		\node[box] (dom1) {\(\{0,1\}^m\times\{0,1\}^n\)};
%		\node[box, right=of dom1] (codom1) {\(\{0,1\}^n\)};
%		\node[box, below=of dom1] (dom2) {\(\{0,1\}^m\)};
%		\node[box, right=of dom2] (codom2) {\(\text{Perm}(\{0,1\}^n)\)};
%		\node[box, below=of codom2] (perm) {\(\text{Perm}(\{0,1\}^n)\)};
%		
%		% Arrows
%		\draw[arrow] (dom1) -- node[midway, above] {\(f\)} (codom1);
%		\draw[arrow] (dom2) -- node[midway, above] {\(f\)} (codom2);
%		
%		% Sampling process
%		\draw[sample] (codom2) -- ++(2, 0) node[right] {\(f_k\)};
%		\draw[sample] (perm) -- ++(2, 0) node[right] {\(P\)};
%		
%		% Uniform sampling
%		\node[left=1.5cm of dom2] (sample1) {\(k \xleftarrow{\$} \{0,1\}^m\)};
%		\node[left=1.5cm of perm] (sample2) {\(P \xleftarrow{\$} \text{Perm}(\{0,1\}^n)\)};
%		\draw[sample] (sample1) -- (dom2);
%		\draw[sample] (sample2) -- (perm);
%		
%		% Indistinguishability
%		\draw[indist] (perm) -- node[midway, right] {Indistinguishability} ++(0, 4.5) coordinate (top) -- ++(-5, 0) coordinate (left) -- ++(0, -4.5) -- (codom2);
%		
%	\end{tikzpicture}
%\end{document}

%\documentclass[tikz,border=3.14mm]{standalone}
%\usepackage{amsmath}
%\usetikzlibrary{positioning,arrows.meta,decorations.pathreplacing}
%
%\begin{document}
%	\begin{tikzpicture}[
%		node distance=1.5cm and 2cm,
%		box/.style={draw, rectangle, minimum width=2.5cm, minimum height=1cm, align=center},
%		arrow/.style={-{Latex}, thick},
%		curly/.style={decorate, decoration={brace, amplitude=10pt, mirror}},
%		perm/.style={draw, rectangle, minimum width=2.5cm, minimum height=1cm, align=center, dashed}
%		]
%		
%		% Nodes
%		\node[box] (cipher) {$f: \{0,1\}^m \times \{0,1\}^n \to \{0,1\}^n$};
%		\node[box, below=of cipher] (currying) {$f: \{0,1\}^m \to \text{Perm}(\{0,1\}^n)$};
%		\node[box, below=of currying] (samplek) {$k \xleftarrow{\$} \{0,1\}^m$};
%		\node[perm, right=of samplek] (permk) {$f_k \in \text{Perm}(\{0,1\}^n)$};
%		\node[perm, right=of permk] (randomperm) {$P \xleftarrow{\$} \text{Perm}(\{0,1\}^n)$};
%		
%		% Indistinguishability label
%		\node[below right=1cm and 2.5cm of samplek] (indist) {
%			Indistinguishability between $f_k$ and $P$
%		};
%		
%		% Arrows
%		\draw[{Latex}-{Latex}, thick] (cipher) -- (currying);
%		\draw[arrow] (currying) -- (samplek);
%		\draw[arrow] (samplek) -- (permk) node[midway, above, sloped] {Currying};
%		\draw[arrow] (permk) -- (randomperm) node[midway, above, sloped] {Comparison};
%		
%		% Curly brace for permutation group
%		\draw[curly] ([yshift=0.5cm] permk.north west) -- ([yshift=0.5cm] randomperm.north east)
%		node[midway, above=10pt] {$\text{Perm}(\{0,1\}^n)$};
%		
%		% Highlight indistinguishability
%		\draw[->, thick, dashed] (permk) to[bend left] (indist);
%		\draw[->, thick, dashed] (randomperm) to[bend right] (indist);
%		
%	\end{tikzpicture}
%\end{document}

\documentclass[tikz,border=2mm]{standalone}
\usepackage{amsmath}
\usetikzlibrary{arrows.meta, decorations.pathmorphing, positioning, calc}

\begin{document}
	\begin{tikzpicture}[
		node distance=2cm and 3cm,
		every node/.style={draw, rectangle, rounded corners, align=center},
		arrow/.style={-{Stealth}, thick},
		decoration={snake, amplitude=.5mm, segment length=3mm},
		dashedarrow/.style={-{Stealth}, dashed, thick}]
		
		% Nodes for elements
		\node[align=center] (blockcipher) {Block Cipher\\$f:\{0,1\}^m \times \{0,1\}^n \to \{0,1\}^n$};
		\node[above right=1cm of blockcipher, align=center] (currying) {Family of Permutations\\ $f:\{0,1\}^m \to\text{Perm}(\{0,1\}^n)$};
		
		% Key sampling node
		\node[below=2cm of currying] (keysampling) {Key Sampling\\$k \xleftarrow{\$} \{0,1\}^m$};
		
		\node[right=of keysampling] (fk) {Fixed Key Function\\$f_k \in \text{Perm}(\{0,1\}^n)$};
		\node[right=of currying] (perm) {Uniform Permutation\\$P \xleftarrow{\$} \text{Perm}(\{0,1\}^n)$};
		
		% Indistinguishability node
%		\node[right=of fk] (indistinguishability) {Indistinguishability\\$f_k \overset{?}{\sim} P$};
		
		
		% Draw arrows
		\draw[arrow, out=0, in=180] (blockcipher.east) to (currying.west);
		\draw[arrow] (currying) -- (keysampling);
		\draw[arrow] (keysampling) -- (fk);
		\draw[arrow] (currying) -- (perm);
		\draw[<->, thick, decorate, decoration={snake, amplitude=1mm, segment length=2mm}, out=90, in=-90] (fk.north) -- (perm.south) node[below right=.5cm] {$f_k\overset{?}{\sim} P$}; %([yshift=0.5cm] perm.north west) -- ([yshift=0.5cm] fk.north east);
%		\draw[arrow] (keysampling) -- (indistinguishability);
%		\draw[arrow] (fk) -- (indistinguishability);
%		\draw[arrow] (perm) -- (indistinguishability);
		
		% Additional annotations
%		\node[below=of perm] (note) {\textit{Indistinguishability between $f_k$ and $P$ under random key $k$}};
		
		% Group action effect representation (curried $f_k$ as action)
		\node[below=1cm of fk] (groupaction) {$f_k(x)$ permutes $\{0,1\}^n$};
		\draw[dashedarrow] (fk.south) -- (groupaction.north);
		
	\end{tikzpicture}
\end{document}
