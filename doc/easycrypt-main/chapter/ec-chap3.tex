% Chapter 1. Installing EasyCrypt

\cite{StoughtonEasyCrypt}

\EasyCrypt is a proof assistant for mechanizing proofs of the security of cryptographic constructions and protocols.

The official EasyCrypt installation instructions are available on the EasyCrypt GitHub. Below is a summary of these instructions that also emphasizes the connection with the Emacs text editor. EasyCrypt can be run from the shell (command line) in batch mode, to check individual .ec files. But when proofs are constructed interactively this is done within Emacs, with the generic interface Proof General mediating between Emacs and EasyCrypt, which is running as a sub-process of Emacs.

These instructions are current for:
\begin{itemize}
	\item version 5.1.1 of the OCaml compiler;
	\item version 1.7.0 of \textsf{why3};
	\item version 2.5.2 of \textsf{alt-ergo}.
\end{itemize}

(EasyCrypt is implemented in OCaml, \textsf{why3} is the interface to SMT solvers used by EasyCrypt, and \textsf{alt-ergo} is one of SMT solvers you will need.) 

%\section{Software Analysis}
%
%\subsection{A Hard Limit}
%
%\subsection{Trade-off}
%
%\section{Basic Principle}
%\subsection{Software Analysis based on Concrete Execution}
%\subsection{Software Analysis based on Symbolic Execution}
%\subsection{Software Analysis based on Abstract Execution}