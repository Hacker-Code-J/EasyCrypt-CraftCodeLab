\section{Attacker Type}
\begin{itemize}
	\item $\coa$; Ciphertext-Only Attack
	\item $\kpa$; Known-Plaintext Attack
	\item $\cpa$; Chosen-Plaintext Attack
	\item $\cca$; Chosen-Ciphertext Attack
	\item $\acca$; Adaptive Chosen-Ciphertext Attack
\end{itemize}
\begin{tikzpicture}[auto, node distance=2cm, >=latex']
	
	% Nodes
	\node [draw, rectangle, rounded corners, text centered, minimum height=2em, minimum width=6em] (easycrypt) {EasyCrypt};
	\node [draw, rectangle, rounded corners, below of=easycrypt, node distance=2cm, text centered, minimum height=2em, minimum width=6em] (why3) {Why3};
	\node [draw, rectangle, rounded corners, below left of=why3, node distance=2.5cm and 1.5cm, text centered, minimum height=2em, minimum width=6em] (altergo) {Alt-Ergo};
	\node [draw, rectangle, rounded corners, below right of=why3, node distance=2.5cm and 1.5cm, text centered, minimum height=2em, minimum width=6em] (z3) {Z3};
	
	% Arrows
	\draw [->] (easycrypt) -- (why3);
	\draw [->] (why3) -- (altergo);
	\draw [->] (why3) -- (z3);
	
	% Labels
	\node [below of=altergo, node distance=1.2cm] (label1) {};
	\node [below of=z3, node distance=1.2cm] (label2) {};
	\node [text width=10cm] at (label1 -| label2) {Integration of Why3, Alt-Ergo, and Z3 within EasyCrypt};
	
\end{tikzpicture}\\


\begin{itemize}
	\item $\mathcal{K}$: key space
	\item $\mathcal{N}$: nonce space
	\item $\mathcal{P}$: plaintext space
	\item $\mathcal{C}$: ciphertext space
\end{itemize}
\[
\texttt{used\_once}(n, \mathcal{N}) = n\in\mathcal{N}.
\]\[
\fullfunction{\texttt{used\_once}}{\mathcal{N}\times\set{\mathcal{N}}}{\set{0,1}}{(n,\mathcal{N})}{n\in\mathcal{N}}\]

\[
\texttt{used\_once}:\mathcal{N}\to[\set{\mathcal{N}}\to\set{0,1}]
\]
%현대 암호학에서는 공격자의 능력치에 따라 공격 유형을 다음과 같이 분류한다.
%1. 암호문 단독 공격()
%− 1개 이상의 암호문이 주어진 상황에서, 각 암호문에 대한 평문을 복원하는 공격
%2. 기지 평문 공격()
%− 동일한 키로 암호화한 평문/암호문 쌍이 1쌍 이상 주어진 상황에서, 동일한 키로 암호화한 새로운 암호문에 대한 평문을
%복원하는 공격
%3. 선택 평문 공격()
%− 공격자가 1개 이상의 원하는 평문에 대한 암호문을 얻은 후, 새롭게 주어진 암호문에 대한 평문을 복원하는 공격
%4. 선택 암호문 공격()12
%− 공격자가 1개 이상의 원하는 암호문에 대한 평문을 얻은 후, 새롭게 주어진 암호문에 대한 평문을 복원하는 공격

\newpage
\section{Initial Vectors}

\begin{itemize}
	\item $\mathcal{K}$: key space
	\item $\mathcal{N}$: nonce space
	\item $\mathcal{P}$: plaintext space
	\item $\mathcal{C}$: ciphertext space
\end{itemize}


\paragraph{Random IV}
Let $p\in\mathcal{P}$ and $k\in\mathcal{K}$. %The probability distribution of $E_k(p,IV)$ over the choice of $IV$ must approximate the uniform distribution over the $\mathcal{C}$, ensuring that:
\[
\Pr\left[E_k(p,IV)=c\right]\approx\frac{1}{\abs{\mathcal{C}}}\quad\text{for all}\ c\in\mathcal{C}.
\]

\paragraph{Nonce IV}
Let $p,q\in\mathcal{P}$ and $n,m\in\mathcal{N}$. \[
\Pr\left[E_k(p,n)=c=E_k(q,m)\right]=0\quad\text{if}\ p\neq q, n\neq m.
\]

\[
\lnot(a\lor b)\iff\lnot a \land \lnot b
\]

\begin{table}[h!]\centering\ttfamily\begin{tabular}{lc}
		split. & $\lnot(a\lor b)\Rightarrow \lnot a \land \lnot b$ \\
		move => not\_or. & $\lnot a \land\lnot b$ \\
		split. & $\lnot a$ \\
		case a. & $a\Rightarrow\lnot\top$ \\
		move => a\_true. & $\lnot\top$ \\
		$\vdots$ & $\vdots$
	\end{tabular}\end{table}
\begin{enumerate}
	\item $\lnot(a\lor b)\Rightarrow\lnot(a\land b)$
	\item $\lnot(a\lor b)\Rightarrow\lnot(a\land b)$
\end{enumerate}
