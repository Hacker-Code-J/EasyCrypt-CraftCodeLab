\section{Prelude}
\subsection{Logic}
\subsubsection*{Principle of Functional Extensionality}
\easycryptcode[linerange={60-60}]{listings/easycrypt/theories/prelude/Logic.ec}
The axiom asserts: \[
f=g\iff f==g
\]
\begin{itemize}
	\item Left-hand side $(f=g)$:
	
	This refers to the equality of functions as mathematical objects. Two functions 
	$f$ and $g$ are equal if they are identical, meaning they are the same function in every aspect.
	\item Right-hand side $(f==g)$:
	
	This refers to pointwise equality: 
	$f(x)=g(x)$ for all $x\in'a$
	\item Interpretation:
	
	The axiom establishes that two functions are equal as mathematical objects if and only if they produce the same output for every input. This is the essence of functional extensionality.
\end{itemize}

\begin{table}[h!]
\begin{tabularx}{\textwidth}{>{\raggedleft\arraybackslash}p{.15\textwidth}|p{.75\textwidth}}
\toprule[1.2pt]
Set Theory & In classical set theory (ZFC), functional extensionality is implicitly satisfied because functions are defined as sets of ordered pairs: \[
f=\set{(x,f(x)):x\in\text{dom}(f)}.
\] Thus, two functions are equal if and only if their values agree for every input. \\ \hline
Category Theory & In category theory, functional extensionality corresponds to the notion that morphisms (arrows) between objects are determined by their action on elements. \\ \hline
Constructive Mathematics & In constructive frameworks (e.g., type theory), functional extensionality may not hold by default, as functions can be defined by their computational behavior rather than just their input-output relations.\par
In such settings, extensionality is often treated as an additional axiom. \\
\bottomrule[1.2pt]
\end{tabularx}
\end{table}

\newpage
\subsubsection*{Constructive Choice Function}
\easycryptcode[linerange={645-658}]{listings/easycrypt/theories/prelude/Logic.ec}
\paragraph{Definition of \texttt{choiceb}} For a predicate $P:\texttt{'a}\to\set{\texttt{true},\texttt{false}}$ and a default element $x_0\in\texttt{'a}$, it selects an element $x\in\texttt{'a}$ such that $P(x)=\texttt{true}$, if such an $x$ exists. Otherwise, it defaults to $x_0$. That is, \[
\texttt{choiceb}(P,x_0)=
\begin{cases}
	x&:\exists x\in\texttt{'a}:P(x)=\texttt{true}\\
	x_0&:\forall x\in\texttt{'a}:P(x)=\texttt{false}\\
\end{cases}
\]
\paragraph{Axioms for \texttt{choiceP}} If there exists an element 
$x$ such that $P(x)=\texttt{true}$, then $\texttt{choiceb}(P,x_0)$
satisfies $P$, \ie, $P(\texttt{choiceb}(P,x_0))=\texttt{true}$.\par
This axiom asserts that the choice function correctly selects an element from the subset defined by $P$, whenever such an element exists. It embodies the constructive aspect of the function.
\paragraph{\texttt{choiceb\_dfl}} If $P(x)=\texttt{false}$ for all $x\in\texttt{'a}$, then 
$\texttt{choiceb}(P,x_0)$ returns the default value $x_0$.\par
This axiom ensures that the function $\texttt{choiceb}$ respects its fallback behavior when the subset defined by $P$ is empty.
\paragraph{Equality of Choices} If two predicates $P$ and $Q$ are logically equivalent (i.e., $P(x)\Leftrightarrow Q(x)$ for all $x\in\texttt{'a}$), then $\texttt{choiceb}(P,x_0)=\texttt{choiceb}(Q,x_0)$.\par
The proof relies on the extensionality of predicates: functions 
$P$ and $Q$ are equivalent if they have identical output for all inputs.
\paragraph{Axiom: Irrelevance of Default for Non-Empty Subsets} If $P$ is satisfied by some element $x$, the value of $\texttt{choiceb}(P,x_0)$ is independent of the default value $x_0$.\par
This axiom guarantees that when $\exists x\in\texttt{'a}:P(x)=\texttt{true}$, the choice function selects an element satisfying $P$, irrespective of the fallback $x_0$. This is because the fallback $x_0$ is only used when $P$ is unsatisfiable.\par
This property reflects the \textbf{stability} of the constructive choice under changes to the fallback parameter, provided the primary condition is satisfied.
\paragraph{Summary} These axioms and lemmas collectively define a constructive choice principle, a central concept in intuitionistic mathematics and proof systems. Unlike classical logic's unrestricted use of the axiom of choice, this constructive version ensures explicit constructability of the chosen element.
\paragraph{Application} This construct is useful in formal verification for:
\begin{itemize}
	\item Selecting elements satisfying a predicate (e.g., in search or optimization problems).
	\item Ensuring robust behavior under different edge cases (e.g., empty or singleton sets).
	\item Proving properties of algorithms that depend on choice.
\end{itemize}

\newpage
\section{Algebra}

\subsection{IntDiv}
\easycryptcode[linerange={17-37}]{listings/easycrypt/theories/algebra/IntDiv.ec}
\paragraph{Quotient (\easycryptinline{\%/})}
\paragraph{Remainder (\easycryptinline{\%\%})}
\paragraph{Divisibility (\easycryptinline{\%|})} $m\ \texttt{\%|} d\iff m\bmod d=0$.
%\paragraph{\texttt{euclidef}} The definition formalizes the Euclidean division, which states: For any integers $m\in\Z$ and $d\neq 0$, there exist unique integers 
%$q,r\in\Z$ such that: \[
%m=q\cdot d+r\quad\text{and}\quad 0\leq r<\abs{d}.
%\] Thus,
%\[
%\texttt{euclidef}(m,d,(q,r))\iff\begin{cases*}
%	m=q\cdot d+r\\
%	0\leq r<\abs{d}&if $d\neq 0$
%\end{cases*}
%\]
%\paragraph{\texttt{edivn}}
\paragraph{Applications} The modular arithmetic constructs are essential for reasoning about cryptographic algorithms (e.g., RSA, Diffie-Hellman).
Division and modulus operations are cornerstones for verifying number-theoretic properties.
\newpage
\easycryptcode[linerange={50-71}]{listings/easycrypt/theories/algebra/IntDiv.ec}
%\begin{tikzpicture}[node distance=2cm, on grid, auto]
%	
%	\node[state, initial] (start) {Start: Rewrite \texttt{edivz\_def}};
%	\node[state] (checkm) [below=of start] {Check \( 0 \leq m \)};
%	\node[state] (posm) [right=of checkm] {Case: \( 0 \leq m \)};
%	\node[state] (negm) [left=of checkm] {Case: \( 0 > m \)};
%	\node[state] (checkd) [below=of checkm] {Check \( d = 0 \)};
%	\node[state] (result) [below=of checkd] {Result: Prove lemma};
%	
%	\path[->]
%	(start) edge node {Rewrite} (checkm)
%	(checkm) edge node {If \( 0 \leq m \)} (posm)
%	(checkm) edge node {Else} (negm)
%	(posm) edge [bend right] node {Validate \( edivnP \)} (checkd)
%	(negm) edge [bend left] node {Validate \( edivnP \)} (checkd)
%	(checkd) edge node {If \( d = 0 \)} (result);
%	
%\end{tikzpicture}

