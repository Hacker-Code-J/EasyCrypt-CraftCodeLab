%\subsection{JUtils}
\easycryptcode[linerange={1-3}]{listings/JUtils.ec}

\begin{table}[h!]
\begin{tabular}{l}
\url{https://github.com/EasyCrypt/easycrypt}\\
easycrypt/theories/core/AllCore.ec\\
easycrypt/theories/core/Bool.ec\\
easycrypt/theories/algebra/IntDiv.ec\\
easycrypt/theories/algebra/StdOrder.ec\\
easycrypt/theories/datatypes/List.ec\\
\end{tabular}
\end{table}

\subsubsection*{LEMMA:\quad \texttt{modz\_comp}}
% modz_cmp
\easycryptcode[linerange={6-7}]{listings/JUtils.ec}
\begin{statement}
	For two integers $m$ and $d>0$, the remainder of 
	$m$ divided by $d$ satisfies:\[
	0\leq m\bmod d< d.
	\]
\end{statement}
\ \\
\begin{analysis}
	This property follows directly from the division algorithm: \[
	m=q\cdot d+r,\quad 0\leq r<d
	\] where $q=\floor*{m/d}$ and $r=m\bmod q$.
\end{analysis}
\ \\
\begin{pftactics}
	SMT solver with the pre-proved property \texttt{edivzP} (in \texttt{IntDiv}).
\end{pftactics}

\newpage
\subsubsection*{LEMMA:\quad \texttt{divz\_cmp}}
% divz_cmp
\easycryptcode[linerange={9-12}]{listings/JUtils.ec}
\begin{statement}
	For integers $d,i,n$ where $d>0$ and $0\leq i< n\cdot d$, the integer division satisfies \[
	0\leq \frac{i}{d}< n.
	\]
\end{statement}
\ \\
\begin{analysis}
	TBA
\end{analysis}
\ \\
\begin{pftactics}
	TBA
\end{pftactics}

\subsubsection*{LEMMA:\quad \texttt{mulz\_cmp\_r}}
% mulz_cmp_r
\easycryptcode[linerange={14-18}]{listings/JUtils.ec}
\begin{statement}
	TBA
\end{statement}
\ \\
\begin{analysis}
	TBA
\end{analysis}
\ \\
\begin{pftactics}
	TBA
\end{pftactics}

\subsubsection*{LEMMA:\quad \texttt{cmpW}}
% cmpW
\easycryptcode[linerange={20-21}]{listings/JUtils.ec}
\begin{statement}
	TBA
\end{statement}
\ \\
\begin{analysis}
	TBA
\end{analysis}
\ \\
\begin{pftactics}
	TBA
\end{pftactics}

\subsubsection*{LEMMA:\quad \texttt{le\_modz}}
% le_modz
\easycryptcode[linerange={23-30}]{listings/JUtils.ec}
\begin{statement}
	TBA
\end{statement}
\ \\
\begin{analysis}
	TBA
\end{analysis}
\ \\
\begin{pftactics}
	TBA
\end{pftactics}