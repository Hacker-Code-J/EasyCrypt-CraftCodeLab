%\documentclass[tikz,border=3mm]{standalone}
%\usepackage{amsmath}
%\usetikzlibrary{arrows.meta, positioning, decorations.pathmorphing}
%
%\begin{document}
%\begin{tikzpicture}[scale=1]
%	\tikzstyle{XOR} = [
%	line width=.25mm,
%	draw,
%	circle,
%	outer sep=2pt,
%	append after command={
%		[shorten >=2bp, shorten <=2bp]
%		(\tikzlastnode.north) edge[line width=.25mm] (\tikzlastnode.south)
%		(\tikzlastnode.east) edge[line width=.25mm] (\tikzlastnode.west)}
%	]
%	
%	\node[] (pt) {Plaintext};  
%	\node[XOR, right=1cm of pt] (xor) {};
%	\node[above=2.5cm of xor] (mask) {$k\xleftarrow{\$}\mathcal{K}$};
%	\node[right=1cm of xor] (ct) {Ciphertext};  
%	\draw[dashed] (-.75,2) rectangle (5.25,4.5);
%	
%	\draw[-{Latex}] (pt) to (xor);
%	\draw[-{Latex}] (xor) to (ct);
%	\draw[-{Latex}] (2.25,2.75) -- ++(0,-.5) -- node[right] {mask} ++(0,-2);
%	
%	%	\node [below=of xor, align=left] {$\enc(k,m)=m\oplus k$\\ $\dec(k,c)=c\oplus k$};
%\end{tikzpicture}
%\end{document}

%\documentclass[tikz,border=3.14mm]{standalone}
%\usepackage{amsmath}
%\usetikzlibrary{positioning,arrows.meta,decorations.pathreplacing}
%
%\begin{document}
%	\begin{tikzpicture}[
%		node distance=1.5cm and 2cm,
%		box/.style={draw, rectangle, minimum width=2.5cm, minimum height=1cm, align=center},
%		arrow/.style={-{Latex}, thick},
%		curly/.style={decorate, decoration={brace, amplitude=10pt, mirror}},
%		perm/.style={draw, rectangle, minimum width=2.5cm, minimum height=1cm, align=center, dashed}
%		]
%		
%		% Nodes
%		\node[box] (cipher) {$f: \{0,1\}^m \times \{0,1\}^n \to \{0,1\}^n$};
%		\node[box, below=of cipher] (currying) {$f: \{0,1\}^m \to \text{Perm}(\{0,1\}^n)$};
%		\node[box, below=of currying] (samplek) {$k \xleftarrow{\$} \{0,1\}^m$};
%		\node[perm, right=of samplek] (permk) {$f_k \in \text{Perm}(\{0,1\}^n)$};
%		\node[perm, right=of permk] (randomperm) {$P \xleftarrow{\$} \text{Perm}(\{0,1\}^n)$};
%		
%		% Indistinguishability label
%		\node[below right=1cm and 2.5cm of samplek] (indist) {
%			Indistinguishability between $f_k$ and $P$
%		};
%		
%		% Arrows
%		\draw[{Latex}-{Latex}, thick] (cipher) -- (currying);
%		\draw[arrow] (currying) -- (samplek);
%		\draw[arrow] (samplek) -- (permk) node[midway, above, sloped] {Currying};
%		\draw[arrow] (permk) -- (randomperm) node[midway, above, sloped] {Comparison};
%		
%		% Curly brace for permutation group
%		\draw[curly] ([yshift=0.5cm] permk.north west) -- ([yshift=0.5cm] randomperm.north east)
%		node[midway, above=10pt] {$\text{Perm}(\{0,1\}^n)$};
%		
%		% Highlight indistinguishability
%		\draw[->, thick, dashed] (permk) to[bend left] (indist);
%		\draw[->, thick, dashed] (randomperm) to[bend right] (indist);
%		
%	\end{tikzpicture}
%\end{document}

\documentclass[11pt, tikz]{standalone}
\usepackage{amsmath}
\usetikzlibrary{arrows.meta, decorations.pathmorphing, positioning, calc}

\begin{document}
\begin{tikzpicture}[scale=1]
	\tikzstyle{XOR} = [
	line width=.25mm,
	draw,
	circle,
	outer sep=2pt,
	append after command={
		[shorten >=2bp, shorten <=2bp]
		(\tikzlastnode.north) edge[line width=.25mm] (\tikzlastnode.south)
		(\tikzlastnode.east) edge[line width=.25mm] (\tikzlastnode.west)}
	]
	
	\node[] (pt) {Plaintext};  
	\node[XOR, right=1cm of pt] (xor) {};
	\node[above=2.5cm of xor] (mask) {$k\xleftarrow{\$}\mathcal{K}$};
	\node[right=1cm of xor] (ct) {Ciphertext};  
	\draw[dashed] (-.75,2) rectangle (5.25,4.5);
	
	\draw[-{Latex}] (pt) to (xor);
	\draw[-{Latex}] (xor) to (ct);
	\draw[-{Latex}] (2.25,2.75) -- ++(0,-.5) -- node[right] {mask} ++(0,-2);
	
	%	\node [below=of xor, align=left] {$\enc(k,m)=m\oplus k$\\ $\dec(k,c)=c\oplus k$};
\end{tikzpicture}
\end{document}
